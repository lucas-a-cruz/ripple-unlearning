\section{Limitations}
\label{sec:limitations}
We also note some limitations of our framework and analysis. Firstly, it is limited by the existing popular benchmarks its supports, which have been regarded as “weak measures of unlearning progress” \citep{thaker2025position}. The setups may not accurately reflect realistic model learning or unlearning dynamics, with the underlying forget-retain paradigm itself warranting further scrutiny \citep{thaker2025position}. There's a clear need for more realistic, yet controlled, fine-grained unlearning benchmark setups beyond the currently popular benchmarks. Secondly, while our meta-evaluation of metrics and comparison of methods is a valuable step, its findings need to be extended to more unlearning setups and unlearning algorithms, to gain a greater understanding of the best and comprehensive ways to quantify unlearning. Finally, while our meta-evaluation focuses on knowledge faithfulness and metric robustness as minimal desiderata, these might not be a comprehensive set of desiderata for good unlearning metrics.

\section{Broader Impact}
\label{sec:broader_impacts}
The widespread deployment of AI systems in domains ranging from conversational assistants and recommendation systems to self‑driving vehicles and medical diagnostics raises important concerns about privacy, safety, and regulatory compliance. As these systems are deeply integrated within society, the ability to remove unwanted or sensitive information from deployed models (“unlearning”) is essential to maintain safety, reliability and uphold legal requirements.

Our work on a unified, extensible LLM unlearning benchmark accelerates progress toward reliable, scalable unlearning solutions. By standardizing implementations of unlearning methods, evaluation metrics, and stress tests across diverse tasks and datasets, we lower the barrier for both academic and industrial adoption. This facilitates rapid iteration on novel techniques, ensures consistent measurement of privacy and utility trade‑offs, and enables model governance workflows that can respond promptly to deletion or correction requests.

In the long run, advances enabled by this framework will support trustworthy AI deployment in safety‑critical and highly regulated settings. From ensuring that autonomous vehicles do not retain outdated or hazardous driving data, to empowering personalized assistants with user‑controlled memory, robust unlearning mechanisms will be a cornerstone of ethical, privacy‑preserving machine learning. By fostering community collaboration and transparent evaluation, our research paves the way for AI systems that adapt responsibly to evolving societal norms and regulatory landscapes.