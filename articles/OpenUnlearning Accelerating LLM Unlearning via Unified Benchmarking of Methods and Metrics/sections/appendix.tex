







\section{Additional details on OpenUnlearning's components}


\subsection{Unlearning benchmarks}
\paragraph{\tofu{}:}  
A synthetic fine-grained knowledge‑unlearning benchmark with 200 fictitious author profiles, each offering 20 QA pairs and a defined “forget set”, and a finetuned chat LLM. \tofu{}’s primary metric is Truth Ratio, which measures the \textit{relative} likelihood of the true answer after unlearning.

\paragraph{\muse{}:}  
A memorization and knowledge unlearning benchmark targeting the removal of books and news articles from a finetuned LLM. MUSE evaluates for memorization (via verbatim reproduction rates), knowledge (via question-answers) and privacy protection (using membership inference attacks).

\paragraph{\wmdp{}:}  
An alignment‑focused benchmark of 3,668 multiple‑choice questions probing hazardous knowledge in biosecurity, cybersecurity, and chemical security, paired with corresponding unlearning corpora and off-the-shelf chat LLMs. WMDP assesses a model’s ability to forget dangerous capabilities while preserving general performance.  

\paragraph{Improvements:}
\label{subsec:improving_benchmarks}
In evaluations, \tofu{} reuses training questions, raising concerns about overfitting and inflated metrics. To mitigate this, we evaluate on paraphrased questions in our meta-evaluation and benchmarking. We also extend \tofu{} with privacy-based metrics from \muse{} via PrivLeak~\citep{shi2025muse} and introduce additional MIA attacks. For this we create new holdout datasets by replicating the original \tofu{} data generation setup.\footnote{We use the same \texttt{gpt-4-1106-preview} endpoint and prompts for data generation.} We add MIA beyond Min-K \cite{shi2023detecting} to \muse{}. Given the poor quality and tokenization issues users faced with the \textsc{Phi}-1.5 and \textsc{Llama}-2 models from \tofu{}, we introduce new starter target models. \ou{} provides three sizes of the recent \llama{}-3 models: 1B, 3B, and 8B, giving users greater flexibility to experiment. 
Additionally, we augment both \tofu{} and \muse{} with metrics such as Extraction Strength~\citep{carlini2021extracting}, Exact Memorization~\citep{tirumala2022memorization}, and Forget Fluency~\citep{mekala-etal-2025-alternate}.
 We integrate \ou{} with LM Eval Harness \citep{eval-harness} to assess general LLM capabilities that identify post-unlearning degradations, in addition to enabling \wmdp{} evaluations. Several contemporary works can further enhance these benchmarks. We plan to continuously improve the framework by adding-and encouraging contributions of-new features and metrics to both existing and future benchmarks, such as the recent work by \citet{thaker2025position}.



\subsection{Datasets}

In machine unlearning, benchmarks typically structure data into two primary components: (1) forget sets, which contain text corpora and queries designed to test whether the model has successfully erased targeted information, and (2) retain sets, which verify that the model preserves unrelated, desirable knowledge. Beyond this fundamental split, unlearning benchmarks often include additional variations to test algorithmic robustness. For example, scaling splits vary the size of the forget set to assess how well algorithms handle larger deletion requests, while topic-based splits examine whether forgetting specific content impacts retention across semantically related or unrelated domains \cite{maini2024tofu, shi2025muse}. These nuanced splits are essential for assessing scalability, generalization, and sustainability of unlearning methods under realistic conditions. 


\begin{figure}[!ht]
\centering
\begin{minipage}[t]{0.45\textwidth}
\centering
\small
\textbf{(a) Dataset Handler}
\begin{boxminted}{Python}
class PretrainingDataset(Dataset):
    def __init__(self, hf_args, ...):
        ...
        
    def __getitem__(self, idx):
        ...
        return item

_register_data(PretrainingDataset)
\end{boxminted}
\end{minipage}%
\hspace{0.02\textwidth}
\begin{minipage}[t]{0.45\textwidth}
\centering
\small
\textbf{(b) Dataset Configuration}
\begin{boxminted}{Yaml}
MUSE_forget:
  handler: PretrainingDataset
  args:
    hf_args:
      path: "muse-bench/MUSE-News"
      name: "raw"
      split: "forget"
    text_key: "text"
    max_length: 2048
\end{boxminted}
\end{minipage}
\caption{Adding a dataset in \ou: (a) the Python handler class implementing data preprocessing and reusable to load several datasets, and (b) the configuration file specifying arguments for instantiating a particular dataset variant. Adding variants of other modules (e.g. unlearning method trainers, benchmarks, evaluation metrics etc.) involves a similar procedure.}
\label{fig:adding_component_dataset}
\end{figure}

\ou{} provides a modular framework where most of the Python implementation for dataset classes is shared across various dataset configurations and benchmarks. It also allows users to define custom dataset classes following the steps presented in \autoref{fig:adding_component_dataset}. We already support three commonly used dataset handlers, each serving a distinct purpose in the unlearning pipeline:

\begin{itemize}[
  leftmargin=10pt,      % no left indent
  labelwidth=*,        % natural label width
  labelsep=0.5em,      % space between label and item text
  itemsep=1pt,         % space between items
  parsep=1pt,          % paragraph separation within items
  topsep=0pt,          % space above and below the list
  partopsep=0pt        % extra space when the list starts a new paragraph
]

% \begin{} 

% --- Dataset bullet (stylistic) ---
\item \texttt{PretrainingDataset}: used for training models on large-scale web corpora; essential for simulating pre-training settings.

\item \texttt{CompletionDataset}: used for evaluating model outputs in a zero-shot or few-shot setting. This format is particularly useful for measuring memorization and information leakage, such as verbatim reproduction of forgotten content. 

\item \texttt{QADataset}: designed for probing models using natural language question-answer interactions, optionally with few-shot examples. This format is critical for assessing whether the model retains or forgets factual knowledge in interactive settings. Moreover, the framework automatically pipelines model-specific input formatting such as including system prompts or special tokens for chat-based models ensuring that queries are executed in a manner consistent with the model’s native interface.

\item \texttt{ForgetRetainDataset}: The unlearning process involves simultaneous optimization on both the forget and retain datasets, requiring concurrent batch loading. This dataset class abstracts this by loading the retain dataset in the same order as the forget dataset for unlearning.
\end{itemize}


\subsection{Metrics}
\label{subsec:metrics_list}

\ou{} supports multiple evaluation metrics and shares common functionalities across metric implementations. Metrics are broadly classified into three categories and summarized below:

% \input{tables/metrics_list}
\paragraph{Memorization Metrics:} These metrics quantify how much the model has memorized information from its training data.

\begin{enumerate}[%
  leftmargin=15pt,      % no left indent
  labelwidth=*,        % natural label width
  labelsep=0.5em,      % space between label and item text
  itemsep=1pt,         % space between items
  parsep=1pt,          % paragraph separation within items
  topsep=0pt,          % space above and below the list
  partopsep=0pt        % extra space when the list starts a new paragraph
]

    \item \textbf{Exact Memorization (EM):} Quantifies memorization by calculating proportion of tokens in the model's response that exactly match those in the ground truth $y$~\citep{tirumala2022memorization}. Formally, it is defined as 
\begin{align}
\text{EM} = \frac{1}{\vert y \vert} \sum_{k} \boldsymbol{1}\left\{ \arg\max_y f(y \mid [x, y^{<k}]; \boldsymbol{\theta}) = y^k \right\},
\end{align}

    \item \textbf{Extraction Strength (ES):}  Quantifies the intensity of memorization by determining the minimal prefix length required to reconstruct the remaining suffix~\citep{carlini2021extracting}.
\begin{align}
\text{ES} = 1 - \frac{1}{\vert y \vert} \min_k \left\{ k \mid f([x, y^{<k}]; \boldsymbol{\theta}) = y^{>k} \right\}.
\end{align}

    % --- Memorization metrics (fix Probability formulas) ---
    \item \textbf{Probability (Prob.):} Directly quantifies the model’s confidence in its output.
    \begin{align}
    \text{Probability} = p\bigl(f(y \mid x)\bigr)
    \end{align}
    
    \item \textbf{Paraphrased Probability (Prob.):} Probability computed on a paraphrased answer $y^{\text{para}}$ to remove template bias.
    \begin{align}
    \text{Para.\ Prob.} = p\bigl(f(y^{\text{para}} \mid x)\bigr)
    \end{align}

    \item \textbf{ROUGE/Paraphrased ROUGE:} Assesses the degree of overlap between the model's output $f(x)$ and the ground truth $y$~\citep{lin-2004-rouge}. This can be computed against many variants of datasets, including paraphrases and jailbreak prompts (next).

    \item \textbf{Jailbreak ROUGE:} To probe for forgotten information, we employ a prefix-based jailbreaking attack by prompting the model with \textit{"Sure, here is the answer:"} (as in \citep{wang2025towardseffective}) and then computing the ROUGE score between the model's response and the ground truth. This metric captures the extent to which suppressed content can still be recovered through prompt manipulation.

    \item \textbf{Truth Ratio:} Measures the model’s preference for the correct answer over a perturbed (incorrect) alternative by comparing their predicted probabilities. A higher value indicates stronger confidence in the correct response. It is defined as:
\begin{align}
\text{Truth Ratio} = \frac{p(y^{\text{para}} \mid x)}{p(y^{\text{para}} \mid x) + p(y^{\text{pert}} \mid x)}
\end{align}
% --- Truth Ratio paragraph (variable-name consistency) ---
where $y^{\text{para}}$ denotes the paraphrased correct answer and $y^{\text{pert}}$ represents an incorrect alternative with similar structure. Note that \citet{maini2024tofu} use a privacy-oriented variant of Truth Ratio computed as $\text{Truth Ratio} = min(\frac{p(y^{\text{para}} \mid x)}{p(y^{\text{pert}} \mid x)}, \frac{p(y^{\text{pert}} \mid x)}{p(y^{\text{para}} \mid x)})$. We modify it so that it quantifies extent of knowledge for our work's purposes.

\end{enumerate}


\paragraph{Privacy Metrics:} These metrics ascertain whether sensitive information from the forget set can still be inferred or extracted from the model. Techniques such as Membership Inference Attacks (MIA) are utilized to evaluate the model's susceptibility to revealing whether specific data points were part of its training set, thereby assessing the privacy guarantees post-unlearning. However, these metrics often assume access to perfectly i.i.d.\ holdout splits or to an “oracle” retain model, limiting their practical usefulness in real-world settings.
\begin{enumerate}[%
  leftmargin=15pt,      % no left indent
  labelwidth=*,        % natural label width
  labelsep=0.5em,      % space between label and item text
  itemsep=1pt,         % space between items
  parsep=1pt,          % paragraph separation within items
  topsep=0pt,          % space above and below the list
  partopsep=0pt        % extra space when the list starts a new paragraph
]
    \item \textbf{MIA:} Evaluates a model’s tendency to memorize training data by testing whether an adversary can distinguish between seen examples from the forget set ($\Db_{\text{forget}}$) and unseen examples from a holdout set ($\Db_{\text{holdout}}$), based on model confidence. Ideally, a model that has not seen the forget set should yield an AUC of 0.5; however, due to challenges in constructing perfect holdout splits, benchmarks such as \muse{} often calibrate this with AUC scores from the retain model (e.g., as done in PrivLeak). We support several MIA methods, including: LOSS~\citep{yeom2018privacy}, ZLib~\citep{carlini2021extracting}, GradNorm~\citep{wang2024pandoraswhiteboxprecisetraining}, MinK~\citep{shi2023detecting}, and MinK++~\citep{zhang2025mink}.
    \item \textbf{Forget Quality:} Performs a statistical test on the truth ratio distributions of the unlearned and retain models, yielding high values when the distributions closely match.
\begin{align}
\text{KS}(\text{Truth Ratio} (f_{\text{target}}, \Db_f), \text{Truth Ratio} (f_{\text{retain}}, \Db_f) )
\end{align}

\end{enumerate}


\paragraph{Utility Metrics:}
The goal of unlearning is to effectively forget the targeted data while preserving the model’s performance on non-forget data. Utility metrics assess whether the model retains its capabilities on broader tasks beyond the retain data, ensuring that unlearning does not degrade general performance on real-world distributions.
\begin{enumerate}[%
  leftmargin=15pt,      % no left indent
  labelwidth=*,        % natural label width
  labelsep=0.5em,      % space between label and item text
  itemsep=1pt,         % space between items
  parsep=1pt,          % paragraph separation within items
  topsep=0pt,          % space above and below the list
  partopsep=0pt        % extra space when the list starts a new paragraph
]
    \item \textbf{Model Utility (MU):} Captures the retained performance of a model after unlearning, both on the closely tied retain set and on broader general knowledge. TOFU computes MU as the harmonic mean of nine metrics across three data levels: the retain set, real authors, and factual world knowledge. At each level, it evaluates three metrics—probability, ROUGE, and the Truth Ratio. 
    item \textbf{ROUGE for knowledge:} \muse{} and \tofu{} assess utility by measuring ROUGE on knowledge-based questions.
    \item \textbf{Forget Fluency:} Prior work \citep{mekala-etal-2025-alternate, gandikota2024erasing} has shown that unlearning often degrades model fluency, particularly on the forget set, resulting in random or nonsensical outputs. To capture this effect, we employ a classifier-based score that predicts whether a given text resembles gibberish\footnote{\url{https://huggingface.co/madhurjindal/autonlp-Gibberish-Detector-492513457}}.
    \item \textbf{LM Eval Harness}: LM Evaluation Harness \citep{eval-harness} is an easy to use library enabling evaluations for a wide variety of general LLM benchmarks. It is integrated into \ou{}, unlocking a broad suite of metrics such as WMDP MCQ, MMLU \citep{hendrycks2021measuringmassivemultitasklanguage}, GSM8K \citep{cobbe2021gsm8k} etc., for comprehensive post-unlearning evaluation.
\end{enumerate}

By integrating the diverse metrics listed in \autoref{tab:openunlearning_components}, \ou{} offers a robust framework to holistically evaluate unlearning methods, ensuring that models not only forget specific data but also maintain utility and privacy standards. \autoref{fig:rouge_metric_example} illustrates the process of adding a new metric to the \ou{} framework.

It is important to recognize that the applicability of unlearning metrics often depends on the dataset used during evaluation. As a result, metrics implemented for one benchmark may not directly transfer to another. For example, the Knowledge Memorization metric in \muse{} is based on question-answer pairs where answers are typically short, single-word responses. In contrast, \tofu{} lacks such a data split and instead features more descriptive, verbose answers. In this context, metrics like ROUGE recall may inadvertently capture surface-level template patterns rather than the core semantic content, potentially misleading the evaluation.

\begin{figure}[!htp]
\centering

\textbf{(a) Metric Handler}
\begin{boxminted}{Python}
@unlearning_metric(name="rouge")
def rouge(model, **kwargs):
    tokenizer = kwargs["tokenizer"]
    data = kwargs["data"]
    collator = kwargs["collators"]
    batch_size = kwargs["batch_size"]
    generation_args = kwargs["generation_args"]
    ... # calculate ROUGE
    return {
        "agg_value": np.mean(rouges),
        "value_by_index": rouges,
    }
\end{boxminted}


\textbf{(b) Metric Configuration}
\begin{boxminted}{Yaml}
# @package eval.muse.metrics.forget_verbmem_ROUGE
defaults: # fill up forget_verbmem_ROUGE's inputs' configs
  - ../../data/datasets@datasets: MUSE_forget_verbmem
  - ../../collator@collators: DataCollatorForSupervisedDatasetwithIndex
  - ../../generation@generation_args: default
handler: rouge # the handler we defined above in (a)
rouge_type: rougeL_f1
batch_size: 8
datasets:
  MUSE_forget_verbmem:
    args:
      hf_args:
        path: muse-bench/MUSE-Books
      predict_with_generate: True
collators:
  DataCollatorForSupervisedDataset: 
    args:
      padding_side: left # for generation
generation_args:
  max_new_tokens: 128
\end{boxminted}

\caption{Example of a metric definition in \ou{}: (a) the Python handler that implements the ROUGE metric, and (b) the corresponding configuration used to run ROUGE-based evaluation for assessing verbatim memorization.}
\label{fig:rouge_metric_example}
\end{figure}


\subsection{Models}

Different language models encode and store knowledge in fundamentally different ways depending on their architecture and training setup. As a result, evaluating unlearning methods across a diverse range of models is essential for assessing their robustness and generalizability. However, existing benchmark implementations often support only a narrow set of model types and require users to manually rewrite evaluation logic such as input formatting, tokenization, and prompting—when adapting to new architectures. For example, chat-based models rely on specialized prompting structures that differ significantly from standard causal language models, making adaptation tedious and error-prone.

\ou{} supports multiple model architectures and sizes out of the box. Built on Hugging Face Transformers \citep{wolf-etal-2020-transformers}, it uses \texttt{AutoModelForCausalLM} and \texttt{AutoTokenizer}, while also supporting custom model loading (e.g., for probe models). A unified abstraction allows seamless switching between chat-style and base models without modifying the unlearning or evaluation pipeline, reducing overhead and enabling consistent cross-model comparisons.

In addition to support loading models in multiple precisions, \ou{} 
also support loading 4-bit and 8-bit quantized models using the \texttt{bitsandbytes} library \citet{dettmers_qlora_bitsandbytes}. This flexibility for quantization is particularly valuable for stress testing unlearning \citet{zhang2025catastrophic}.



\begin{table}[ht]
  \caption{Supported LLM Architectures in \ou{}}
  \label{tab:supported-llms}
  \centering
  \begin{tabular}{ll}
    \toprule
    \textbf{Model} & \textbf{Reference} \\
    \midrule
    \llama-2             & \citet{touvron2023llama} \\
    \llama-3.1 / 3.2     & \citet{grattafiori2024llama} \\
    \textsc{Phi}-1.5             & \citet{li2023textbooksneediiphi15} \\
    \textsc{Phi}-3.5             & \citet{abdin2024phi} \\
    \textsc{Gemma}              & \citet{team2024gemma} \\
    \textsc{Zephyr}             & \citet{tunstall2024zephyr} \\
    \textsc{Qwen-2.5}             & \citet{qwen2025qwen25technicalreport} \\
    \bottomrule
  \end{tabular}
\end{table}


\paragraph{New models for \tofu{}}: \ou{} provides trained models for the \tofu{} benchmark using  \llama-based architectures finetuned on the \tofu{} dataset. These models span a range of sizes including 1B, 3B, and 8B parameters, enabling users to explore unlearning behavior across different model capacities. The 1B model, in particular, offers a highly efficient option for rapid experimentation with turnaround time of 15 minutes, requiring only 20 GB of GPU VRAM.


\begin{figure}[ht]
\centering

\textbf{(a) \llama~3.2 1B model configuration}
\begin{boxminted}{yaml}
model_args:
  pretrained_model_name_or_path: "meta-llama/Llama-3.2-1B-Instruct"
  attn_implementation: 'flash_attention_2'
  torch_dtype: bfloat16
tokenizer_args:
  pretrained_model_name_or_path: "meta-llama/Llama-3.2-1B-Instruct"
template_args:
  apply_chat_template: True
  system_prompt: You are a helpful assistant.
  date_string: 10 Apr 2025
\end{boxminted}

\vspace{1em}

\textbf{(b) \llama~2-7B model configuration}
\begin{boxminted}{yaml}
model_args:
  pretrained_model_name_or_path: "meta-llama/Llama-2-7b-hf"
  attn_implementation: 'flash_attention_2'
  torch_dtype: bfloat16
tokenizer_args:
  pretrained_model_name_or_path: "meta-llama/Llama-2-7b-hf"
template_args: 
  apply_chat_template: False
  user_start_tag: "Question: "
  user_end_tag: "\n"
  asst_start_tag: "Answer: "
  asst_end_tag: "\n\n"
\end{boxminted}

\caption{Example model configurations for two different \llama~variants: (a) \llama~3.2-1B with chat template prompting, and (b) \llama~2-7B with manual prompt formatting.}
\label{fig:model_config_comparison}
\end{figure}


\subsection{Unlearning Methods}
\label{subsec:unl_methods}

Unlearning methods form the core of the \ou{} framework. In practice, researchers proposing new unlearning approaches often evaluate them on a single benchmark due to the high efforts of adapting their code to other frameworks. This fragmentation has led to a lack of comprehensive, cross-benchmark comparisons in the unlearning literature. The overhead of re-implementing methods, adapting to different evaluation pipelines, and aligning metrics discourages reproducibility and slows progress.


\ou{} addresses this gap by providing a unified and modular infrastructure that abstracts away benchmark-specific details. Researchers can implement their method once, typically by extending a custom \texttt{Trainer}, and instantly evaluate it across multiple benchmarks. This design dramatically lowers the barrier to method development, evaluation and encourages the community to develop robust methods that work across benchmarks.
We currently support all commonly used baselines as well as several state-of-the-art methods, and we invite the community to build upon this foundation.

\paragraph{Gradient Ascent \citep{maini2024tofu}:} Performs gradient ascent on the forget set to degrade model confidence on targeted data.
\begin{equation}
    \Lb = -\gamma\mathbb{E}_{(x,y_{\mathrm{f}})\sim\Db_{\text{forget}}}\ell\big(y_{\mathrm{f}}|x;f_{\text{unl}}\big)
\end{equation}

\paragraph{GradDiff \citep{maini2024tofu}:} Performs gradient ascent on forget data and descent on retain data.
\begin{align*}
\Lb = -\gamma\mathbb{E}_{(x,y_{\mathrm{f}})\sim\Db_{\text{forget}}}\ell\big(y_{\mathrm{f}}|x;f_{\text{unl}}\big) +\alpha\mathbb{E}_{(x,y)\sim\Db_{\text{retain}}} \ell\big(y|x;f_{\text{unl}}\big)
\end{align*}

\paragraph{IdkNLL \citep{maini2024tofu}:} Trains to output "I don't know" responses when queried on forgotten content.
\begin{align*}
\Lb = \gamma\mathbb{E}_{(x,y_{\mathrm{f}})\sim\Db_{\text{forget}}}\ell\big(y_{\mathrm{idk}}|x;f_{\text{unl}}\big)+\alpha\mathbb{E}_{(x,y)\sim\Db_{\text{retain}}} \ell\big(y|x;f_{\text{unl}}\big)
\end{align*}

\paragraph{IdkDPO \citep{maini2024tofu}:} Uses a DPO-style objective to align the model to output "I don't know" responses when queried on forgotten content.
\begin{align*}
\Lb = -&\frac{2}{\beta}\mathbb{E}_{(x,y_{\mathrm{f}})\sim\Db_{\text{forget}}}\log \sigma \Big(-\beta \log\Big(\frac{p(y_{\mathrm{idk}}|x;f_{\text{unl}})}{p(y_{\mathrm{idk}}|x;f_{\text{target}})}\Big)-\beta \log\Big(\frac{p(y_{\mathrm{f}}|x;f_{\text{unl}})}{p(y_{\mathrm{f}}|x;f_{\text{target}})}\Big)\Big) \\
&+\alpha\mathbb{E}_{(x,y)\sim\Db_{\text{retain}}} \ell\big(y|x;f_{\text{unl}}\big)
\end{align*}

\paragraph{NPO \citep{NPO_zhang2024negative}:} Similar to the DPO-style objective, but uses only the negative feedback term in its formulation. It demonstrates better training stability compared to similar methods like GradDiff. \\
\begin{align*}
\Lb = -&\frac{2}{\beta}\mathbb{E}_{(x,y_{\mathrm{f}})\sim\Db_{\text{forget}}}\log \sigma \Big(-\beta \log\Big(\frac{p(y_{\mathrm{f}}|x;f_{\text{unl}})}{p(y_{\mathrm{f}}|x;f_{\text{target}})}\Big)\Big) \\
&+\alpha\mathbb{E}_{(x,y)\sim\Db_{\text{retain}}} \ell\big(y|x;f_{\text{unl}}\big)
\end{align*}

\paragraph{SimNPO \citep{fan2024simplicity_simnpo}:} A modified variant of NPO that retains its core forgetting behavior \textbf{by} replacing the reference model with $\delta$ in the loss formulation.

\begin{align*}
\Lb = -\frac{2}{\beta}\mathbb{E}_{(x,y_{\mathrm{f}})\sim\Db_{\text{forget}}}\log \sigma \Big(-\frac{\beta}{|y_{\mathrm{f}}|} \log p(y_{\mathrm{f}}|x;f_{\text{unl}}) - \delta \Big)\Big) +\alpha\mathbb{E}_{(x,y)\sim\Db_{\text{retain}}} \ell\big(y|x;f_{\text{unl}}\big)
\end{align*}

\paragraph{AltPO \citep{mekala-etal-2025-alternate}:} Uses a DPO-style objective to align the model toward generating alternate, in-domain plausible facts (produced by the model itself) that introduce ambiguity and suppress the original target knowledge.
\begin{align*}
\Lb = -&\frac{2}{\beta}\mathbb{E}_{(x,y_{\mathrm{f}})\sim\Db_{\text{forget}}}\log \sigma \Big(-\beta \log\Big(\frac{p(y_{\mathrm{alt}}|x;f_{\text{unl}})}{p(y_{\mathrm{alt}}|x;f_{\text{target}})}\Big)-\beta \log\Big(\frac{p(y_{\mathrm{f}}|x;f_{\text{unl}})}{p(y_{\mathrm{f}}|x;f_{\text{target}})}\Big)\Big) \\
&+\alpha\mathbb{E}_{(x,y)\sim\Db_{\text{retain}}} \ell\big(y|x;f_{\text{unl}}\big)
\end{align*}


\paragraph{RMU \citep{li2024wmdp}:} Assumes knowledge is encoded in model parameters and manipulates these representations to suppress memorization signals for the forget set while preserving knowledge in the retain set. Let $\boldsymbol\phi(s;f_{\text{unl}})$ denote the embedding features of the model, the loss is given by
\begin{align*}
    \Lb = &\mathbb{E}_{(x,y_{f})\sim\Db_{\text{forget}}} \frac{1}{\vert y_{\mathrm{f}}\vert}\sum_{i=1}^{\vert y_{\mathrm{f}}\vert}\vert\vert\boldsymbol{\phi}([x,y^{<i}];f_{\text{unl}})-c\cdot\boldsymbol{u}\vert\vert_2^2 \nonumber \\
    &+\mathbb{E}_{(x,y)\sim\Db_{\text{retain}}} \frac{1}{\vert y\vert}\sum_{i=1}^{\vert y\vert}\vert\vert\boldsymbol{\phi}([x,y^{<i}];f_{\text{unl}})-\boldsymbol{\phi}([x,y^{<i}];f_{\text{target}})\vert\vert_2^2,
    \label{eq: rmu}
\end{align*}
where $\boldsymbol{u}$ has elements randomly sampled from $[0, 1)$ and $c$ is a scaling hyper-parameter. 

\paragraph{UNDIAL \citep{dong-etal-2025-undial}:}
Mitigates the instability found in prior methods by employing self-distillation, where the model learns from its own adjusted outputs. The core idea is to reduce the model's confidence in the target token by adjusting its logits, thereby diminishing its influence without affecting the overall model performance. This is achieved by minimizing the KL divergence between the adjusted logits and the model's current output distribution.
\begin{align*}
z_{\text{adj}}(x) &= z_{\text{orig}}(x) - \beta \cdot \mathbf{1}_{y_f} \\
\Lb = \gamma \mathbb{E}_{(x, y_f) \sim \mathcal{D}_{\text{forget}}} & \left[ \text{KL}\left( \text{softmax}(z_{\text{adj}}(x)) \,\|\, \text{softmax}(z_{\text{unl}}(x)) \right) \right] +\alpha\mathbb{E}_{(x,y)\sim\Db_{\text{retain}}} \ell\big(y|x;f_{\text{unl}}\big)
\end{align*}

Where $z_{\text{orig}}(x)$ is the original logits produced by the model before unlearning and  $z_{\text{adj}}(x)$ is the adjusted logits. 
%This method notably achieves a good ranking in our benchmarking \S\autoref{sec:leaderboard}.


\subsection{Technical improvements:}
\label{appsec:tech_improvments}
\paragraph{Efficiency:}
\muse{} evaluates models without batching, while our implementation uses batched inference to improve efficiency. \tofu{} pads all sequences to a fixed \texttt{max\_length} of 512, resulting in unnecessary GPU memory and compute overhead. In contrast, we apply dynamic padding based on the longest sequence in each batch. \wmdp{} lacks a rigorous training and unlearning framework, limiting its extensibility for developing and evaluating new methods.

\paragraph{Training paradigms supported:} Training or unlearning with larger models (e.g., $\geq$ 8B parameters) presents a significant computational challenge, often necessitating multiple high-end GPUs such as NVIDIA A100s. To accelerate this process, we support:
\begin{enumerate}[%
  leftmargin=10pt,      % no left indent
  labelwidth=*,        % natural label width
  labelsep=0.5em,      % space between label and item text
  itemsep=1pt,         % space between items
  parsep=1pt,          % paragraph separation within items
  topsep=0pt,          % space above and below the list
  partopsep=0pt        % extra space when the list starts a new paragraph
]
    \item \textbf{DeepSpeed ZeRO Stage-3} \cite{jacobs2023ulysses}: Enabled via the Accelerate library \cite{accelerate}, reducing the memory usage through optimizer state partitioning and CPU/NVMe offloading.
    \item \textbf{Model Parallelism}: Splits the model across GPUs along its layers, allowing large models to be trained even when individual GPUs cannot hold the full model in memory.
\end{enumerate}

\section{Experimental setup}
\label{appsec:exp_setup}
All subsequent meta-evaluation and benchmarking experiments use the \llama{}-3.2-1B model. Experiments use BF16 precision, a single NVIDIA A100 GPU, a batch size of 32 and a paged AdamW optimizer (matching the \tofu{} paper's default settings).

\section{Meta-evaluation}
\label{appsec:meta_eval_more_details}

\subsection{Faithfulness test-bed design} 
\label{appsec:faithfulness_setup}
We create two pools of models: the negative $N$ and the positive $P$ pool. $N$ contains models trained with varying training parameters while avoiding the knowledge of the forget set in the training data. $P$ contains models trained similarly to $N$ but with the target knowledge included in training. During the model pool preparation, we modify the training data used in the $N$ and $P$ pools with several training data variants. This introduces model diversity, forcing metrics to detect genuine \textit{knowledge} retention rather than non-knowledge related artifacts, to achieve high scores. The faithfulness evaluation pipeline is illustrated in \autoref{fig:meta_eval} (a).

\begin{enumerate}[leftmargin=12pt,itemsep=4pt]
  \item \textbf{Positive pool ($P$):} Models are trained on all TOFU facts (both \texttt{forget10} and \texttt{retain90}). We then replace \texttt{forget10} with two transformed variants. First, \texttt{forget10\_paraphrased} uses paraphrased labels while preserving factual content. Second, \texttt{forget10\_bio} contains long-form biographies derived from \texttt{forget10}.
  \item \textbf{Negative pool ($N$):} Models are trained on the \texttt{retain90} split of \tofu{}, along with two perturbed variants of \texttt{forget10}. First, \texttt{forget10\_perturbed} pairs each forget prompt with an incorrect label. Second, \texttt{celeb\_bio} (biographies of random celebrities) serves as the counterpart to \texttt{forget10\_bio}.
\end{enumerate}

To further diversify the model pool, we vary training hyperparameters: five learning rates from $1\times10^{-5}$ to $5\times10^{-5}$, and two checkpoints (after training epochs 5 and 10). Combining 2 pools $\times$ 3 dataset variants $\times$ 5 learning rates $\times$ 2 checkpoints yields 60 models in total.

\paragraph{Data generation process}  
While some of TOFU’s evaluation datasets include paraphrased and perturbed examples, our training‑set variants for the model pool were generated independently. We used \llama{} 3.1 405B via the SambaNova API\footnote{\url{https://cloud.sambanova.ai/playground}} to paraphrase and perturb QA pairs, and prompted Gemini\footnote{\texttt{gemini-2.0-flash-exp} (accessed 26 April 2025)} to produce Wikipedia‑style biographies from each author’s 20 QA pairs.

\subsection{Robustness setup design} We create a large and diverse pool of
unlearned models and a separate set of retain models, which serve as gold-standard references having never been trained on the forget set. The unlearned pool is then subjected to stress‑test interventions, to provoke recovery (or inducing) of the forgotten knowledge. These pools serve as our test-bed. For every metric being meta-evaluated, values are recorded on both pools before and after each intervention. The change in a metric's distribution before and after intervention on the unlearned models (along with the change in retain models for normalization) is used to characterize robustness. We use three interventions: \textit{relearning}, \textit{quantization} and \textit{probing}. 

\begin{enumerate}[leftmargin=12pt,itemsep=4pt]
    \item \textbf{Relearning Setup:} We finetune the unlearned model on the full \texttt{forget10} dataset for one epoch with a learning rate of \(2\times10^{-5}\).
    \item \textbf{Quantization Setup:} We apply 4-bit floating-point quantization using BitsAndBytes \citep{dettmers_qlora_bitsandbytes}. Checkpoints unlearned with a learning rate of \(1\times10^{-5}\) are chosen, as quantization is most effective at lower learning rates \citep{zhang2025catastrophic}.
    \item \textbf{Probing:} We evaluate layer 11 of the \llama{}-3.2-1B model (16 layers total) using the language-model head from the corresponding \texttt{retain90}-trained model. This head is trained with a learning rate of \(1\times10^{-4}\) on \texttt{retain90} for ten epochs.
\end{enumerate}

\subsection{Additional Results}

\autoref{fig:faithfulness_all} shows the faithfulness of the metrics, while \autoref{fig:relearn_all} and \autoref{fig:quantization_all} show their behavior under relearning and quantization stress tests. We found that removing MU filter of retaining at least 80\% utility for unlearned models reduces robustness to quantization further (see \autoref{fig:quantization_all_nomu}). Despite this, we apply the MU filter to better align with common unlearning reporting practices.

\paragraph{Probing results:}
\label{subsec:probe}
We compute the metric robustness to probing intervention as follows
\begin{equation}
p = \frac{m^a_{\text{ret}}}{m^a_{\text{unl}}} \quad \text{if} \quad \frac{m^b_{\text{ret}}}{m^b_{\text{unl}}}\geq1, \quad P = \min(p,1)
\end{equation}
\autoref{tab:probing_score} shows the results of our metric meta‐evaluation with probing. 
Probing, while provided for by \ou{}, is not used in the meta-evaluation procedure, as $P$ scores on \tofu{} achieve 1 for all metrics and thus offer little information.

\begin{table}
  \caption{Robustness meta-evaluation with probing (layer 11)}
  \label{tab:probing_score}
  \small
  \centering
  \begin{tabular}{lc}
    \toprule
    \textbf{Metrics} & \textbf{Probe} $\uparrow$ \\
    \midrule
    Exact Mem. & 1.0 \\
    Extr. Strength & 1.0 \\
    Truth Ratio & 1.0 \\
    Prob. & 0.99 \\
    ROUGE & 0.99 \\
    Jailbreak ROUGE & 0.99 \\
    Para. Prob. & 1.0 \\
    Para. ROUGE & 0.99 \\
    MIA - LOSS & 1.0 \\
    MIA - MinK & 1.0 \\
    MIA - MinK++ & 0.83 \\
    MIA - ZLib & 1.0 \\
    \bottomrule
  \end{tabular}
\end{table}

\subsection{Further considerations}
\label{subsec:meta_eval_further_considerations}
\paragraph{Why aren't the intervened versions of metrics considered evaluation metrics themselves?} The interventions we use require modification to and access of model weights, which an unlearning auditor might not possess. In the case of relearning and quantization, they also involve computational costs associated with training and calibration. Stress-testing interventions are best suited for final-stage audits before model deployment, rather than for routine use throughout unlearning workflows, as is
expected of standard evaluation metrics. Our analysis can inform the design of robust evaluation metrics that function without requiring stress-testing. 

\paragraph{Comparison to \citet{wang2025towardseffective}'s meta-evaluation:}
Our work is related to the recent effort by \citet{wang2025towardseffective} to compare unlearning evaluation metrics. Their analysis focuses on four metrics: probability, ROUGE, ES, and EM, and evaluates robustness by measuring the linear correlation of metric values before and after applying stress-tests such as jailbreaking, relearning, probing, and token noising. We extend this framework in several key ways. 

\vspace{-5pt}
\begin{enumerate}[leftmargin=12pt,itemsep=0pt]
    \item \textbf{Broader metric coverage:} We evaluate a broader range of metrics, including six additional ones.
    
    \item \textbf{Faithfulness assessment:} We assess faithfulness of metrics in our meta-evaluation as a minimal criterion. This enforces that good metrics must accurately capture the presence or absence of target knowledge, rather than merely resisting change under intervention.
    
    \item \textbf{Focused interventions:} We focus specifically on three interventions: relearning, probing, and quantization, excluding jailbreaking and token-noising from the intervention set. We instead treat jailbreaking as an evaluation metric in its own right. Prompt-based attacks like paraphrasing and jailbreak-style prompts are more naturally seen as inexpensive evaluation metrics rather than stress-testing interventions. Additionally, \citet{wang2025towardseffective} found jailbreaking and token noising (which is also a prompt modification) to be less effective as interventions. 
    
    \item \textbf{A different calibration criterion:} Our procedure also introduces a calibration criterion grounded in ideal behavior. Rather than expecting linear variation from a metric upon intervention, we benchmark metric behavior against a gold-standard retain model, for a more principled signal of robustness.
    
    \item \textbf{Practical robustness analysis:} Our robustness analysis filters for models with good utility that are substantially unlearned, selected from a diverse and representative set of unlearning algorithms. This leads to a test distribution for metrics that better reflects realistic unlearning scenarios.
\end{enumerate}

\floatplacement{figure}{H}
\begin{figure}
  \centering
  \includegraphics[width=0.9\linewidth]{images/faithfulness_all.pdf}
  \caption{Faithfulness: Evaluation of multiple metrics to assess faithfulness. AUC indicates how effectively metrics distinguish between models trained on the target knowledge and those that are not.}
  \label{fig:faithfulness_all}
\end{figure}

\begin{figure}
  \centering
  \includegraphics[width=0.9\linewidth]{images/relearn_all.pdf}
  \caption{Relearning: Stress-testing multiple evaluation metrics through relearning. A significant fraction of unlearned models regain knowledge faster than the retained model when re-exposed to the forgotten data, falling into the unreliable red-shaded region: indicating that the metrics failed to initially capture the knowledge and are thus not robust.}
  \label{fig:relearn_all}
\end{figure}

\begin{figure}
  \centering
  \includegraphics[width=0.9\linewidth]{images/quantization_all.pdf}
  \caption{Quantization: Stress-testing multiple evaluation metrics through quantization. For several metrics, a subset of unlearned models shows increased metric values after quantization, falling into the red-shaded region: suggesting that the metrics failed to initially capture the presence of knowledge and are therefore not robust. These results are reported only for models unlearned with low learning rates and high utility.}
  \label{fig:quantization_all}
\end{figure}


\begin{figure}
  \centering
  \includegraphics[width=0.9\linewidth]{images/quantization_all_nomu.pdf}
  \caption{Quantization: Stress-testing multiple evaluation metrics through quantization. For each metric, a subset of unlearned models shows increased metric values after quantization, falling into the red-shaded region, suggesting that the metrics failed to initially capture the presence of knowledge and are therefore not robust. These results are reported only for models unlearned with low learning rates and no filter on utility.}
  \label{fig:quantization_all_nomu}
\end{figure}

\newpage
\section{Further discussion on benchmarking unlearning methods}
\label{appsec:leaderboard}

\subsection{Metric aggregation}
\label{subsec:benchmark_metric_agg}

There are theee dimensions evaluated by our suite of metrics 1) Memorization, 2) Privacy 3) Utility.

We consider multiple metrics in each dimension and aggregate the score as follows:

\begin{enumerate}[leftmargin=*, itemsep=4pt, label=\arabic*.]
    \item \textbf{Memorization}: To quantify the degree of successful forgetting, the Memorization Score is calculated as the Harmonic Mean (HM) of 4 core metrics which are best as per our meta-evaluations analysis in \S\ref{tab:meta_eval} --- ES, EM, Paraphrased Probability  and Truth Ratio. These metrics are inverted (i.e., $1 - \text{metric}$) so that higher scores indicate more effective unlearning. The score is given by:
    $$ \text{Memorization Score} = \text{HM}\left(1 - \text{ES},\ 1 - \text{EM},\ 1 - \text{Para. Prob},\ 1 - \text{Truth Ratio}\right) $$

    \item \textbf{Privacy}: For assessing privacy, we utilize four Membership Inference Attack (MIA) metrics: LOSS, ZLib, Min-k, and Mink++. For each of these, an individual privacy score ($s_{\text{MIA}}$) is calculated. This score, ranging from 0 to 1, quantifies how closely the unlearned model's behavior on the specific MIA metric aligns with that of a gold-standard retain model (details below). A higher $s_{\text{MIA}}$ score indicates greater similarity to the retain model. The overall Privacy Score is then the Harmonic Mean (HM) of these individual scores:
    $$ \text{Privacy Score} = \text{HM}(s_{\text{LOSS}}, s_{\text{ZLib}}, s_{\text{Min-k}}, s_{\text{Mink++}}) $$


    \item \textbf{Utility}: TOFU evaluates a model’s utility using nine core metrics that assess performance across splits at three different distances from the forget dataset distribution - namely, retain, real-world authors, and wrong-fact queries: using QA probability, ROUGE, and truth-ratio scores. In addition to this we include a new metric that measures the fluency of the model's response when prompted with entities-related to forget queries, following \citep{mekala-etal-2025-alternate, gandikota2024erasing}. Fluency is assessed using a classifier\footnote{\url{https://huggingface.co/madhurjindal/autonlp-Gibberish-Detector-492513457}} that detects gibberish / nonsensical outputs. The final utility score is the harmonic mean of MU and fluency. Note that we scale all metrics with init finetuned model, so their scores across all points fall in the $[0, 1]$ range. For example, TOFU MU scores never exceed that of the initial target model upon unlearning, so all scores are effectively divided by the target model's MU.
    
\end{enumerate}

Note that for many metric aggregations we use Harmonic Mean, as HM ensures that a high final score demands strong performance in all constituent parts.

% \begin{center}
% \begin{minipage}{0.28\textwidth}
% \begin{equation*}
% \text{Deviation} = |m_{\text{ret}} - m_{\text{unl}}|
% \end{equation*}
% \end{minipage}
% \hfill
% \begin{minipage}{0.38\textwidth}
% \begin{equation*}
% \text{Norm} =
% \begin{cases}
% m_{\text{ret}} & \text{if } m_{\text{ret}} > m_{\text{unl}} \\
% 1 - m_{\text{ret}} & \text{otherwise}
% \end{cases}
% \end{equation*}
% \end{minipage}
% \hfill
% \begin{minipage}{0.28\textwidth}
% \begin{equation*}
% m\text{-score} = 1 - \frac{\text{Deviation}}{\text{Norm}}
% \end{equation*}
% \end{minipage}
% \end{center}

\subsection{Hyperparameter tuning and model selection while comparing unlearning methods}

\label{subsec:tuning_details}

\paragraph{Hyperparameters used}
\begin{enumerate}[%
  leftmargin=15pt,      % no left indent
  labelwidth=*,        % natural label width
  labelsep=0.5em,      % space between label and item text
  itemsep=1pt,         % space between items
  parsep=1pt,          % paragraph separation within items
  topsep=0pt,          % space above and below the list
  partopsep=0pt        % extra space when the list starts a new paragraph
]
\item For GradDiff and IdK-NLL: we vary the learning rate over the set $\{1\times10^{-5}, 2\times10^{-5}, 3\times10^{-5}, 4\times10^{-5}, 5\times10^{-5}\}$, and sweep the regularization coefficient $\alpha \in \{1, 2, 5, 10\}$.
\item For IdK-DPO, NPO and AltPO: we tune learning rates in $\{1\times10^{-5}, 2\times10^{-5}, 5\times10^{-5}\}$, and search over $\alpha \in \{1, 2, 5\}$ and $\beta \in \{0.05, 0.1, 0.5\}$.
\item For RMU: we use the same learning rate range $\{1\times10^{-5}, 2\times10^{-5}, 5\times10^{-5}\}$, vary the steering coefficient in $\{1, 10, 100\}$, and apply the loss at one of the layers $l \in \{6, 11, 16\}$ of the LLama3.2-1B model. For each selected layer $l$, we restrict training to layers $l-2$, $l-1$, and $l$. 
\item For SimNPO: we tune learning rates in $\{1\times10^{-5}, 2\times10^{-5}, 5\times10^{-5}\}$, and search over $\beta \in \{3.5, 4.5\}$, $\delta \in \{0, 1\}$ and $\delta \in \{0.125, 0.25\}$.
\item For UNDIAL: we tune learning rates in $\{1\times10^{-5}, 1\times10^{-4}, 3\times10^{-4}\}$, and search over $\alpha \in \{1, 2, 5\}$ and $\beta \in \{3, 10, 30\}$.
\end{enumerate}




We aggregate utility score and memorization score and use their harmonic mean for tuning the models. 

\paragraph{What metrics are appropriate for model selection during hyperparameter tuning?} The nature of tuning in unlearning benchmarking has distinct considerations compared to general machine learning. While standard machine learning avoids using test data for tuning to ensure generalization, unlearning in \tofu{} and \muse{} specifically targets the known forget set for erasure. Consequently, iteratively refining the unlearning by evaluating the model's behavior concerning this specific set is a permissible approach to ensure thorough forgetting before deployment. For this tuning, we advocate relying on metrics realistically available during the development phase, specifically those assessing forget quality on the target data and general utility, while avoiding ``oracle" metrics that presume access to unavailable resources like true i.i.d holdout sets or retain models like in \citep{maini2024tofu, shi2025muse}. Since all our privacy scores use a retain model, we avoid them during tuning. We rely on the harmonic mean of the Memorization and Utility scores as the validation objective. 


\paragraph{Comparison to \citet{wang2025towardseffective}'s benchmarking:} While \citet{wang2025towardseffective} propose approaches towards model selection and benchmarking through validation on Extraction Strength and calibration via model-merging, their analysis has several limitations. They rely only on ES scores for evaluating forgetting and utility. ES was found to be robust among the set of 4 evaluation metrics (an observation also re-verified in our work (\S\ref{sec:meta_eval}). Yet it has not been proved that ES is a comprehensive metric validating all facets of knowledge unlearning. For example, ES does not account for privacy metrics that prevent over-unlearning, like \tofu{}'s Truth Ratio or FQ or \muse{}'s PrivLeak. In addition, they do not consider all facets of general utility evaluation, particularly forget set fluency. Finally, the question of what metrics can be used in model selection and if they must be separate from the leaderboard metrics remains unanswered. These limitations remain, to a smaller degree, in our benchmarking procedure, and we consider this an important line for further research.

\begin{table}
  \caption{\small
  Comparison of unlearning methods on the \tofu{} task, showing aggregate (Agg.) using only Memorization (Mem.) and utility (Utility) scores. Privacy scores are not used in the aggregation and are only shown for illustration. Higher scores indicate better performance ($\uparrow$). Initial finetuned is the target model before unlearning and Retain model is the gold standard target model. The focus on memorization as opposed to privacy results in GradDiff performing the best as it easily results in over-unlearning.}
  \label{tab:leaderboard_with_mem_mu}
  \centering
  \begin{tabular}{lcccc}
    \toprule
    \textbf{Method} & \textbf{Agg.} $\uparrow$ & \textbf{Mem.} $\uparrow$ & \textbf{Priv.} $\uparrow$& \textbf{Utility} $\uparrow$ \\
    \midrule
    Init. finetuned & 0.00 & 0.00 & 0.10 & 1.00 \\
    Retain & 0.58 & 0.31 & 1.00 & 0.99 \\
    \midrule
    GradDiff \citep{maini2024tofu}  & \textbf{0.87} & \textbf{0.97} & 3.27e-03 & 0.79 \\
    AltPO \citep{mekala-etal-2025-alternate} & \underline{0.76} & \underline{0.63} & 0.06 & \underline{0.95} \\
    IdkDPO \citep{maini2024tofu} & 0.71 & 0.56 & 0.06 & \underline{0.95} \\
    NPO \citep{NPO_zhang2024negative} & 0.69 & 0.52 & 0.06 & \textbf{0.99} \\
    RMU \citep{li2024wmdp} & 0.53 & 0.47 & 0.5 & 0.61 \\
    SimNPO \citep{fan2024simplicity_simnpo} & 0.49 & 0.32 & \underline{0.63} & 1.0 \\
    UNDIAL \citep{dong-etal-2025-undial} & 0.4 & 0.27 & 0.48 & 0.78 \\
    IdkNLL \citep{maini2024tofu} & 0.14 & 0.08 & 0.17 & 0.93 \\
    \bottomrule
  \end{tabular}
\end{table}



